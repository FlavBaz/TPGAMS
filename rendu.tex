\documentclass[11pt,a4paper]{article}
\usepackage[utf8]{inputenc}
\usepackage[T1]{fontenc}
\usepackage[french]{babel}
\usepackage{geometry}
\usepackage{booktabs}
\usepackage{amsmath,amssymb}
\usepackage{graphicx}
\usepackage{hyperref}
\usepackage{xcolor}
\usepackage{enumitem}

\geometry{margin=2cm}
\setlength{\parindent}{0pt}
\setlength{\parskip}{0.5em}

\title{\textbf{Planification du pompage dans un réseau d'eau potable}\\
\large Projet d'optimisation MINLP -- Réseau 4 Tanks}
\author{Flavien BAZON}
\date{\today}

\begin{document}
\maketitle


% %==============================================================================
% \section{Description du problème et modélisation}
% %==============================================================================

% Le réseau \textit{4 Tanks} est composé de 7 nœuds (1 source, 2 jonctions, 4 réservoirs), 3 pompes en parallèle et 6 canalisations. L'objectif est de minimiser le coût électrique du pompage sur un horizon de 24h avec tarification jour/nuit.

% \subsection{Variables de décision}
% \begin{itemize}[noitemsep]
%     \item $x_{k,t} \in \{0,1\}$ : état de la pompe $k$ à l'instant $t$
%     \item $q^{pompe}_{k,t} \geq 0$ : débit de la pompe $k$ (m³/h)
%     \item $q^{pipe}_{n,n',t} \geq 0$ : débit dans la canalisation $(n,n')$ (m³/h)
%     \item $h_{n,t}$ : charge hydraulique au nœud $n$ (m)
%     \item $v_{r,t}$ : volume d'eau dans le réservoir $r$ (m³)
% \end{itemize}

% \subsection{Contraintes principales}
% Conservation des débits, équilibre offre-demande, limites de volume, contraintes de perte de charge (quadratiques), contraintes de gain de charge des pompes (bilinéaires $h \cdot x$).

%==============================================================================
\section{Modèles implémentés}
%==============================================================================

\begin{table}[h]
\centering
\begin{tabular}{@{}lccc@{}}
\toprule
\textbf{Modèle} & \textbf{Type} & \textbf{Convexité} & \textbf{Variables} \\
\midrule
MINLP complet & Quadratique & Non-convexe & Mixte (binaires + continues) \\
Sans pression & Linéaire & Convexe & Mixte (binaires + continues) \\
Relaxation convexe & Quadratique & Convexe & Mixte (binaires + continues) \\
\bottomrule
\end{tabular}
\caption{Typologie des trois modèles implémentés}
\label{tab:typologie}
\end{table}

\textbf{MINLP complet :} Modèle avec toutes les contraintes hydrauliques, incluant les pertes de charge quadratiques et les contraintes bilinéaires $h(s,t) \cdot x(k,t)$ pour le gain de charge des pompes.

\textbf{Sans pression :} Relaxation où les contraintes de charge hydraulique sont supprimées.

\textbf{Relaxation convexe :} Les contraintes non-convexes (égalités bilinéaires) sont remplacées par des inégalités convexes.

Modèles implémentés pour le 4WT et le customer network


%==============================================================================
\section{Résultats numériques}
%==============================================================================

\subsection{Démarche et première analyse}

J'ai d'abord testé le modèle 4Tanks MINLP complet en résolvant avec BARON sur le serveur ECOS, sur la 2e run j'ai rajouté une limite de temps de 300s pour comparer avec les résultats du tableau de référence.
Les résultats obtenus sur ces 2 runs sont dans le tableau ci-dessous.



Essayons ensuite avec DICOPT, on obtient:

\begin{table}[h]
\centering
\begin{tabular}{@{}lcccc@{}}
\toprule
\textbf{Run} & \textbf{Coût (€)} & \textbf{Borne inf. (€)} & \textbf{Temps (s)} & \textbf{Remarque} \\
\midrule
BARON (8h) & 11.0006 & 9.8446 & 28\,902 & Solution MINLP valide \\
BARON (300s) & 11.4363 & 9.8446 & 300 & Solution MINLP valide \\
DICOPT (300s) & 9.9833* & 9.9030* & 300 & *MIP linéarisé uniquement \\
DICOPT (8h) & 9.9833* & 9.9197* & 28\,930 & *MIP linéarisé uniquement \\
\midrule
\textit{Référence} & \textit{11.28} & -- & \textit{300} & -- \\
\bottomrule
\end{tabular}
\caption{Résultats BARON/DICOPT. *Les valeurs DICOPT correspondent au MIP linéarisé et ne sont pas des solutions réalisables du MINLP.}
\label{tab:runs_minlp}
\end{table}

Qui sont les solveurs utilisés ?
Le premier est BARON (pour Branch-And-Reduce Optimization Navigator), c'est un solveur global (donc il va trouver la meilleure solution). Pour cela il va procéder à une relaxation convexe puis utiliser une évolution de l'algorithme de branch and bound: le branch and reduce, qui partionne l'espace sur des variables bianires et continues !
BARON garantit de trouver l'optimum global si on lui laisse assez de temps. 

Ensuite on a DICOPT, pour DIscrete and Continuous OPTimizer, utilise la méthode Outer Approximation (approximation d'une fonciton non linéaire à l'aide de ses tangentes) pour résoudre le MILNP. Il ne résout rien lui-même mais divise le problème et coordonne deux sous-solveurs : Un solveur NLP (CONOPT) et un solveur MIP (CPLEX). Pour faire ça l'algo procède par Itérations (résolution du NLP après relaxation continue -> linéarisation des contraintes non linéaires autour de la solution, ie création des tangentes -> résoudre le MIP pour fixer les entiers -> une fois les entiers fixés résoudre le NLP, ajouter une tangente -> répéter jusqu'à convergence).
Mais DICOPT ne garantit qu'un optimum local, de plus l'OA marche moins bien pour des fonctions non convexes et la tangente peut alors exclure la solution.

Les résultats obtenus avec BARON confirment la validité du modèle MINLP. Avec un temps de calcul de 8 heures, BARON atteint une solution à 11.0006€. Le run limité à 300 secondes trouve une solution à 11.4363€, ce qui reste acceptable. La différence entre ces deux runs illustre l'importance du temps alloué pour les problèmes MINLP non-convexes : BARON améliore progressivement sa solution au fil des itérations du Branch-and-Reduce.
On observe que la borne inférieure de BARON (9.8446€) reste significativement éloignée de la solution trouvée. Ce phénomène est caractéristique des problèmes non-convexes : la relaxation convexe sous-estime le vrai coût car elle autorise des solutions qui ne respectent pas exactement les contraintes bilinéaires. Fermer ce gap nécessiterait un temps de calcul très important au vu de la lenteur de montée de la borne inf.

Les résultats DICOPT nécessitent une interprétation plus prudente. Contrairement à BARON, DICOPT n'utilise pas de Branch-and-Bound sur le MINLP et ne produit donc pas de bornes supérieure/inférieure au sens classique. Comme on l'a vu, il procède par Outer Approximation : les contraintes non-linéaires sont linéarisées autour de points de fonctionnement successifs, puis le MIP résultant est résolu par CPLEX.

Dans notre cas, DICOPT n'a jamais complété une seule itération de l'algorithme. La résolution du MIP linéarisé par CPLEX a consommé la totalité du temps de calcul (  Cplex Time: 28930.89sec ) sans atteindre l'optimalité. Les valeurs 9.9833€ et 9.9197€ reportées dans le tableau correspondent donc aux bornes du MIP linéarisé, pas du MINLP original.

Pour ce problème MINLP non-convexe avec contraintes bilinéaires, BARON s'avère le solveur approprié. DICOPT ne fournit pas de solutions exploitables.

revoir le 2 et 3, meme resultats et trop bas pour le 3
4 pas de resultats ?

\subsection{Tableau}
Pour le 4 tanks:
\begin{table}[h]
\centering
\begin{tabular}{@{}lcccccc@{}}
\toprule
\textbf{Modèle} & \textbf{Solveur} & \textbf{Coût (€)} & \textbf{Borne inf.}  & \textbf{Temps (s)} \\
\midrule
MINLP complet & BARON & 11.44 & 9.8446 & 300 \\
Sans pression & CPLEX & 7.37 & &  /~2s \\
Relax. convexe & BARON & \textcolor{blue}{À compléter} & &  \\
\midrule
\multicolumn{2}{@{}l}{\textit{Référence (sujet)}} & 11.28 / 9.73 / 10.99 & -- &  300 / 30 / 60 \\
\bottomrule
\end{tabular}
\caption{Résultats obtenus pour le réseau 4 Tanks}
\label{tab:resultats}
\end{table}

Pour le customer network :

\end{document}